\documentclass{article}

% Set page size and margins
% Replace `letterpaper' with `a4paper' for UK/EU standard size
\usepackage[letterpaper,top=2cm,bottom=2cm,left=3cm,right=3cm,marginparwidth=1.75cm]{geometry}
\usepackage[spanish]{babel}
\usepackage{url}
\usepackage{hyperref}
\usepackage{csquotes}
\usepackage{amsmath}
\usepackage{amssymb}
\usepackage{graphicx}
\usepackage{listings}
\usepackage{xcolor}
\usepackage{setspace}

\graphicspath{ {assets/} }

%Code colors:
\definecolor{codegreen}{rgb}{0,0.6,0}
\definecolor{codegray}{rgb}{0.5,0.5,0.5}
\definecolor{codepurple}{rgb}{0.58,0,0.82}
\definecolor{backcolour}{rgb}{0.95,0.95,0.92}
\doublespacing

\lstdefinestyle{mystyle}{
    backgroundcolor=\color{backcolour},
    commentstyle=\color{codegreen},
    keywordstyle=\color{magenta},
    numberstyle=\tiny\color{codegray},
    stringstyle=\color{codepurple},
    basicstyle=\ttfamily\footnotesize,
    breakatwhitespace=false,
    breaklines=true,
    captionpos=b,
    keepspaces=true,
    numbers=left,
    numbersep=5pt,
    showspaces=false,
    showstringspaces=false,
    showtabs=false,
    tabsize=2
}

\lstset{style=mystyle}

\NewDocumentCommand{\codeword}{v}{%
    \texttt{\textcolor{black}{#1}}%
}

\begin{document}

\begin{titlepage}
    \centering
    {\includegraphics[width=0.5\textwidth]{logo2}\par}
    {\bfseries\LARGE Universidad Católica del Uruguay \par}
    \vspace{0.3cm}
    {\scshape\Large Facultad de Ingeniería \par}
    \vspace{0.3cm}
    {\scshape\Huge Proyecto \\Tarea 4 - Titanik \par}
    \vspace{1cm}
    {\Large Probabilidad y Estadística Aplicada \par}
    {\Large Profesor: Maglis Mujica \par}
    \vfill
    {\Large Autores: \par}
    {\Large Martín Caraballo (5.303.799-7)\\Piero Saucedo (5.342.503-5)\\Juan Martín Riccetto (5.324.939-0)\\Juan Manuel Perez () \par}
    \vfill
    {\Large \today \par}
\end{titlepage}

\section{Introducción}\label{sec:introduccion}
    En el marco teórico de Álgebra Aplicada, se propone a los estudiantes una tarea centrada en el uso de vectores.

    En esta ocasión, el objetivo principal es, dado un número de 15 frases seleccionadas, armar un modelo
    vectorial en base a palabras “clave”, que se vean repetidas a lo largo de las mismas, además estas palabras
    claves son clasificadas dentro de los grupos “positivas”, “neutrales” y “negativas”, de forma que sea posible
    estudiar distintas características de las frases mencionadas.

\section{Marco Teórico}\label{sec:marco-teorico}
    Este marco teórico pretende proporcionarle al lector una compresión de los conceptos utilizados durante el
    desarrollo de la tarea, por lo que se abarcarán aquellos pertinentes a la misma.

\subsection{Vector}\label{subsec:vector}
    Un vector es una entidad matemática que tiene magnitud y dirección.
    Es una herramienta fundamental en matemáticas y física, utilizada para representar cantidades que no solo tienen
    un tamaño (magnitud), sino también una dirección.

    Un vector con un origen fijado queda determinado a partir de dos elementos:
    \begin{itemize}
        \item Una \textbf{semirrecta} partir de dicho origen, es decir, una dirección hacia la que apunta.
        \item Un número no negativo, llamado \textbf{módulo} del vector y que mide su tamaño.
    \end{itemize}

    Alternativamente, se puede fijar un sistema de coordenadas del espacio \textit{n} \textit{n-dimensional};
    entonces un vector queda unívocamente determinado mediante \textit{n} números, llamados coordenadas del vector.

\subsection{Matriz}\label{subsec:matrix}
    
\end{document}